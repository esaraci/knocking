\documentclass[a4paper,10pt]{scrartcl}
\usepackage[utf8]{inputenc}

\usepackage{amsmath}
\usepackage{amsfonts}
\usepackage[english]{babel}
\usepackage{fontenc}

% Added packages
\usepackage{graphicx}
\usepackage{xcolor}
\usepackage{booktabs}

% CUSTOM COMMANDS
\newcommand\todo[1]{\textcolor{red}{#1}}
\newcommand\TODO[1]{\textcolor{red}{#1}}
\newcommand\eve{\textit{Eve}}
\newcommand\alice{\textit{Alice}}


% Title Page
\title{Digital Forensics - Project Report}
\subtitle{Project 1 - Packet classification for mobile applications}
\author{Eugen Saraci - 1171697\\ Università degli Studi di Padova \\\texttt{\small{eugen.saraci@studenti.unipd.it}}}

\begin{document}

\maketitle
\begin{abstract}
 In the last years the large diffusion \todo{qualche altro termine non sarebbe male} of SSL/TLS has made it harder for attackers to collect clear text information through packet sniffing or, more in general, through network traffic analysis. The main reason for this is that SSL/TLS encrypts the traffic between two endpoints, which means that even though packets can still be easily captured, no useful information can be inferred from the packet's content without having the encryption keys.\footnote{It is worth mentioning that the endpoints of the communication (i.e. source and destination IP addresses) are transmitted in clear text for routing purposes; by performing a DNS lookup of the addresses and attacker could easily infer what site a user is visiting.}

The authors of \cite{contiknocking} and \cite{contianalysis} showed that by training a machine learning algorithm with encrypted traffic data, one could correctly classify which actions a user actions performed on the most common Android applications such as Facebook, Gmail, or Twitter. This could easily lead, through a correlation attack, to the full deanonimization of fake, privacy preserving identities.

In this work I try to reproduce the results achieved in \cite{contiknocking} and \cite{contianalysis} by implementing the classification model described in the papers.
\vspace{0.5cm}
\end{abstract}

\section{Introduction and Notation}
\label{sec:intro}

\subsection{Actions and Flows}
\label{subsec:flows}
In order to easily understand this work, in the next two subsections I introduce two general concept used throughout this report.

\subsubsection{Action and Action Label}
An \textit{action} is simply the action performed by a user while using one of the aforementioned Android apps. Examples of actions are: clicking on a profile page, tweeting a message, sending an email etc. Please note that ``clicking on a profile page'' is what I refer to as the \textit{action label}, in many cases I use the words ``action'' or ``action flow'' to refer to the set of flows that represent that action. \todo{spiegare meglio sta roba}

\subsubsection{Flows}
When a user performs an action some encrypted packets are exchanged with the destination server. A flow consists of the sequence of the byte sizes of the exchanged packets. If the packet is going from the user's phone to the server it is said to be \textit{outgoing}; if the packet is coming from the server to the user's phone it is said to be \textit{incoming} and it is marked with a ``-'' sign before the integer number representing its size.\footnote{The ``-'' sign is just notation, packets cannot have a negative size.} An example of a 5 packet flow is: \texttt{[-12, 80, 90, -111, 30]}. Please note that a single action performed by the user usually generates multiple flows of different dimensions, by that follows that an action actually consists of multiple flows. The techniques used by the authors to determine which flows belong to which action, the ordering of the packets, the packet capturing system, the packet filtering system, and the statistical analysis on the flows will not be treated in this report since the starting point for this work comes when the dataset is already constructed.

\subsection{Notation}
\label{subsec:notation}
\begin{itemize}
 \item $ A $: an action; it represents a sequence of flows;
 \item $ a $: action label;
 \item $ F $: a flow; it represents a sequence of packets;
 \item $ p $: a single packet, it is an integer number representing the size in bytes of that packet.
\end{itemize}

Please note that all of the above can be subscripted by indexes; a subscripted element means that that element is the i-th element of a sequence, e.g. $F_i$ is the $i$-th flow of a sequence of flows (possibly an action $A$).
 
By this follows that $A_i = [F_1, \dots, F_n]_i$ and $F_i = [p_1,\dots, p_m]_i$. Note that $n$ and $m$ are possibly (and probably) different for each flow $F_i$ and for each action $A_i$, even for two actions $A_i, A_j$ where $a_i = a_j$.

\subsection{Dataset}
The dataset consists of 252,151 rows (samples) and 12 columns (features), moreover, the data collected contains packets of different actions for 7 different Android applications: Facebook, Twitter, Gmail, Google Plus, Tumblr\footnote{Because of some inconsistencies between the papers and the dataset, I could not provide a classifier for the Tumblr's actions.}, Dropbox, and Evernote. 

In Table \ref{tab:origdataset} we can see the format of the dataset. I have purposively hidden some of the columns since I do not use them in this work. In the table we can see that each row contains the features of a single flow; to decide which flows belong to which action we compare the \textit{sequence\_start} field which is a fake but consistent timestamp of when the user started that action; When two rows have the same timestamp we can safely assume that they belong to the same action.


\begin{table}[]
\label{tab:origdataset}
\begin{tabular}{@{}lllll@{}}
\toprule
\textbf{action\_start}      & \textbf{app}               & \textbf{action\_label}     & \textbf{\dots}             & \textbf{flow}                \\ \midrule
1383129102.11              & facebook                   & open facebook              & \dots                      & [-15, 75, 144]               \\
1383129102.11              & facebook                   & open facebook              & \dots                      & [-55, -255, -333, 122, -55]  \\
\multicolumn{1}{c}{\dots} & \multicolumn{1}{c}{\dots} & \multicolumn{1}{c}{\dots} & \multicolumn{1}{c}{\dots} & \multicolumn{1}{c}{\dots}   \\
1383129102.11              & facebook                   & open facebook              & \dots                      & [12, 12, 155, 155, -18, 255] \\
1383129244.01              & facebook                   & click menu                 & \dots                      & [78, -206]                   \\
\multicolumn{1}{c}{\dots} & \multicolumn{1}{c}{\dots} & \multicolumn{1}{c}{\dots} & \multicolumn{1}{c}{\dots} & \multicolumn{1}{c}{\dots}   \\ \bottomrule
\end{tabular}
\caption{\small{Some rows of the dataset.}}
\end{table}

\section{Machine Learning}
\label{sec:ml}


\subsection{Dynamic Time Warping - INGLOBARE NEL CAPITOLO MACHINE LEARNING}
\textbf{Dynamic Time Warping} or \textbf{DTW}, is an algorithm used to measure similarity between two time series even if they are of different lengths and/or have repetitions or deletions in them. If we view every single flow $F$ as a time series of packets $p_i,\dots p_m$, we can use \textbf{DTW} to measure how similar two flows are. The reason we are interested in this will become clear later.

\subsection{Machine Learning}
Ideal goal: we want to develop a machine learning algorithm that outputs the \textit{action label} $a_i$ given the \textit{action's flows} $A_i = [F_1,\dots, F_n]$.

Given the fact that each action generates multiple flows of different lengths, we know that standard supervised learning approaches are hard to apply. To see why standard approaches would not work we need to think about the structure of the input and output spaces. Our output space i.e. what we want to predict would be the \textit{action label} $a_i$, while our input space, i.e. the predictors, would be the flows generated by $a_i$ which we denote as $[F_1,\dots,F_n]_i $. One way we could represent flows as features would be to have a feature for each flow $F_j$ generated by the action $a_i$, and the value of a feature would be the sequence of byte sizes of the flow $[p_1,\dots, p_m]_j$. Because of the different number of flows of each action ($n$) we would immediately see that each row could possibly have a different number of features. The main problem of this approach is that we are artificially defining features with no real justification; in other words, we have no reason to associate the first flow of an action with the first flow of another action by marking them as the first feature of the respective samples.

To overcome this obstacle the original authors applied a two stage process. In the first stage, by using an unsupervised clustering algorithm they addressed their missing knowledge about the number of flows and their features. The aim of the first stage is to identify some features for our actions. The second stage exploits the just found features to perform a the canonic classification.
\subsubsection{Hierachical Agglomerative Clustering}
In the typical clustering scenario the goal of the algorithm is to find $k$ groups called clusters, where the \textit{inter-cluster} similarity is very high while the \textit{intra-cluster} similarity is minimal, meaning that the samples belonging to one clusters are very similar to each other while still being very different from samples of other clusters. Notice that the number of clusters $k$ and the similarity function between samples $f\_dist()$ have to be explicitly defined by the teacher. The output of a clustering algorith \todo{finire spiegazione}

In my specific implementation $k = 50$ and $f_dist = dtw$.
In my specific implementation I use as samples all the flows by themselves, ignoring temporarily the idea of action. The reason for this is that w
\subsubsection{Random Forest}

\section{Evaluation}
\label{sec:eval}
\subsection{Experimental Setup}
For this work I implemented two scripts using Python 3.6.5. The two scripts represent the two stages of clustering and classification shown in Figure~\ref{fig:actual}; the first script, named \texttt{clustering.py}, performs the clustering and saves the intermediate representation as a .csv file called \texttt{[appname]\_dataset.csv}. This file is then read as input by \texttt{classifier.py} which will consequently perform the classification using the Random Forest algorithm presented in Section~\ref{subsubsec:rfc}. The computed performance measures will be printed to \texttt{stdout}.

It has to be noted that I trained a model for each different application, this means that the model that classifies Facebook actions has no way to classify Tumblr actions since none of the Tumblr flows has been shown to it during the clustering process or during the Random Forest training phase.

Most of the actions collected in the initial dataset are deemed to be not privacy-violating by the original authors; I do share their opinion, for this reason, before training the Random Forest, I renamed all the labels of the non relevant actions as \textit{other}. This operation is done in memory by the \texttt{classifier.py} script, in this way the intermediate dataset stays untouched and many different configuration of relevant actions can be tried. The list of relevant action for each app can be found inside \texttt{classification.py}.


\subsubsection{Computational Issue}
Because of some issues with the functions provided by the \textit{SciPy} library, I had to keep in memory the distance matrix of the flows. In the distance matrix $D$ each entry $(i,j)$ contains the value of $dtw(i, j)$ which makes $D$ symmetric, which reduces by half the memory needed for the matrix since I could just compute the triangular upper (or lower) part of $D$. Unfortunately this ``optimization'' was not enough; the amount of memory needed was way higher that what my Google Cloud machine had. For this reason I decided to reduce the dataset to about 10,000 flows per application; again this is done in memory by the \textit{clustering.py} script, therefore the original dataset is not modified. After the clustering phase, the generated intermediate dataset contains a number of samples (actions) varying from 2000 to 5000 based on which application is being analyzed. One intermediate dataset is generated for each application.

\todo{qui cito i praametri/input/output di entrambi gli algoritmi}
\subsection{Experimental Results}
\todo{le confusion matrix, precision recall AP}

% ['dropbox', 'evernote', 'facebook', 'gmail', 'gplus', 'tumblr',
%        'twitter'], dtype=object), array([48462, 17014, 50319,  9924, 33471, 56702, 36259]))

% actions in original dataset
% Actions for dropbox: 15103
% Actions for evernote: 6214
% Actions for facebook: 12468
% Actions for gmail: 5644
% Actions for gplus: 17573
% Actions for tumblr: 14989
% Actions for twitter: 7334

% Actions for dropbox: 33
% Actions for evernote: 32
% Actions for facebook: 41
% Actions for gmail: 38
% Actions for gplus: 38
% Actions for tumblr: 47
% Actions for twitter: 25



\begin{thebibliography}{99}
\bibitem{contiknocking}Mauro Conti, Luigi V. Mancini, Riccardo Spolaor, and Nino Vincenzo Verde. 2015. Can't You Hear Me Knocking: Identification of User Actions on Android Apps via Traffic Analysis. In Proceedings of the 5th ACM Conference on Data and Application Security and Privacy (CODASPY '15). ACM, New York, NY, USA, 297-304. DOI: https://doi.org/10.1145/2699026.2699119
\bibitem{contianalysis} Conti, M., Mancini, L. V., Spolaor, R., \& Verde, N. V. (2016). Analyzing android encrypted network traffic to identify user actions. IEEE Transactions on Information Forensics and Security, 11(1), 114-125.

\end{thebibliography}

\end{document}
